\documentclass[11pt]{article}

\usepackage{upquote}

\title{\LaTeX\ Cheatsheet}
\author{Ashfaqur Rahman}
\date{May 12, 2018}

\begin{document}

\maketitle
\tableofcontents
\newpage

A latex cheatsheet.

\section{Characters and Commands}

\subsection{Latex special characters}

\$ \% \& \# \_ \{ \} \verb=\ ~ ^= are special characters and has special meaning to latex.
You can use a leading \textbackslash\ to use them. To print \$, write \textbackslash\$

\subsection{Latex commands}

Each latex command starts with a \textbackslash and contains only letters.
Example: \verb=\maketitle=, \verb=\tableofcontents= etc.\ Parameters to
a command are specified inside \{\} and optional parameters are specified
inside [ ]. For example \verb=\documentclass[10pt]{article}=.

\section{Writing}

\subsection{Modes in \LaTeX}

\begin{description}
	\item[Paragraph mode] Normal mode
	\item[Math Mode] For mathematical equation. Text inside \verb=\(...\)= or \verb=$...$= or \verb=$$...$$= or \verb=\begin{equation}...\end{equation}= or \verb=\begin{displaymath}...\end{displaymath}=
	\item[Left to right mode] Text are displayed from left to right without line breaks. text inside \verb=\mbox{}=
\end{description}

\subsection{Creating a New Paragraph}

A blank line creates a new paragraph.

\subsection{Spaces}

\subsubsection{Inserting Spaces}

\verb=\,= is used for inserting a space.

\subsubsection{Space after a period}

Latex assumes end of sentence if it found period after small case later. In that case it puts extra space after period. But for the cases like ``etc.'' a space followed by \verb=\= should be used after period like \verb=etc.\ =. A space followed by \verb=\=  means inter word space.

If ending of a sentence contains a uppercase letter to end the sentence use \verb=\@= before period.

\subsubsection{Space after a latex command}

All spaces are ignored after a latext command like \verb=\LaTeX= command. To make a space after \LaTeX\ a space followed by \verb=\= should be used like \verb=\LaTeX\ =. Another way to do this is
to use empty \{\} after a command like \verb=\LaTeX{}=.

\subsection{Dashes}

\begin{itemize}
		\item\verb=-= A intra-word dash or hyphene. Example: X-ray.
		\item\verb=--= A number range dash. Example: 1--10
		\item\verb=---= A punctuation dash. Example: ---
\end{itemize}

\subsection{Quotations}

\verb=`= is used for single quotation start (`).\\
\verb='= is used for single quotation end (').\\
\verb=``= is used for double quotation start (``).\\
\verb="= is used for double quotation end (").

\subsubsection{Putting Quotation in a Quotation}

\verb=``\,`Ah!`\,`` She said= produces ``\,\,`Ah!'\,\,'' She said

\subsection{Commenting}

\verb=%= is used for commenting. Latex ignore all character after \verb=%= to the end of the line. Also ignores space in the beginig of the next line.

\section{\LaTeX Commands List}

\begin{itemize}
		\item\verb=\documentclass[]{}= --- To specify document type. Arguments can be book, article, report etc. Options\footnote{Options are specified inside the square brackets} are:

		\begin{enumerate}
			\item 11pt --- 10\% larger than 10pt.
			\item 12pt --- 20\% larger than 10pt.
			\item twoside --- to print in both sides.
			\item twocolumn --- two column in a single page.
		\end{enumerate}

		\item\verb=\usepackage{}= --- To use external packages.
		\item\verb=\title{}= --- To specify document title.
		\item\verb=\author{}= --- To specify documnet author. Multiple authors cab be specified using \verb=\and= command.
		\item\verb=\begin{}= --- To begin any block. like \verb=\begin{document}= to begin any document, \verb=\begin{quote}= begins a long quotation, \verb=\begin{math}=begins a math block etc.
		\item\verb=\end{}= --- ends any block that begins with \verb=\begin{}=
				\item Argument of \verb=\begin= and \verb=\end= are called environments. They create different display environments. Every declaration or command can be environment like \verb=\emph= can also be written as environment like \verb=\begin{emph}= ... \verb=\end{emph}= Here is a list of different environments:

			\begin{description}
				\item[document] Document environment.
				\item[quote] Shor quotation environment.
				\item[quotation] Long quotation environment.
				\item[itemize] Unordered list environment.
				\item[enumerate] Ordered list environment.
				\item[description] Description list environment like this one.
				\item[displaymath] Display math equations in separate line but without equation number.
				\item[equation] Displayequation in separate line with equation number for further referance.
			\end{description}

		\item\verb=\maketitle= --- To create title. It must be inside \verb=\begin{document}=.
		\item\verb=\part= --- Create parts of a document. Doesn't effect documnet numbering.
		\item\verb=\chapter= ---- Create chapters. No applicable for article document class.
		\item\verb=\section= --- Specify a section.
		\item\verb=\subsection= --- Specify a sub-section of a section.
		\item\verb=\subsubsection= --- Specify a sub-section of a sub-section.
		\item\verb=\appendix= --- Specify an appendix.
		\item\verb={}= --- Curly braces can be used for defining scope. Example --- Input: \verb=Hello, {\em aagontuk}= Output: Hello, {\em aagontuk}
		\item\verb=\today= --- Current date \today
		\item\verb=\TeX= --- Shows \TeX
		\item\verb=\LaTeX= --- Shows \LaTeX
		\item\verb=\ldots= --- Produces  \ldots
		\item\verb=\emph{}= --- To emphasize text. Example: I will \emph{emphasize} it.
		\item\verb=\mbox{TEXT}= --- Never breaks TEXT.\ 
		\item\verb=\footnote{footnote}= --- To create a footnote. Example: Footnote\footnote{Please find the foornote here}
		\item\verb=\\= --- New line.
		\item\verb=\\*= --- New line but prevents a new page.
		\item\verb=\newline= --- New line.
		\item\verb=\hfill \break= --- New line.
		\item\verb=\newpage= --- New Page.
		\item\verb=\hspace{xmm}= --- x mm horizontal space.
		\item\verb=\vspace{xmm}= --- x mm vertical space.
\end{itemize}
\end{document}
